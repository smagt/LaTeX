\documentclass[11pt]{article}
\usepackage[a4paper,left=2.5cm,right=2.5cm,bottom=2.8cm,top=2.8cm]{geometry} 
\usepackage[english,british]{babel}

% add colours.  The table option includes the \cellcolor{} commands.
\usepackage[table]{xcolor}
\usepackage{hyperref}
% long wide tables which look good.  We actually do not
% need longtabu this time, since the table spans only one page.
\usepackage{longtable,tabu}
% better spacing in tables
\usepackage{booktabs}
% stuff for the last big table
\usepackage{spreadtab,numprint}

\begin{document}
\section*{Code factory spreadsheet}
%
% the following table is complex.  It uses spreadtab, which can be used to do 
% calculations, and numprint to format the numbers.  Because of spreadtab,
% all text entries must be preceded by @; because of numprint, all text
% entries must be enclosed in {}.  The "n70" entries in the table description
% mean: a numeric field of width 7 before and 0 after the comma.
%
% we use tabu, which we actually do not need for this example. Tabular
% would have done just as well (changing the X in the argument to l
% and changing \taburowcolors1{sum .. sum} to \rowcolor{sum}
%
% for the numprint package: we separate 1000s with a ,
\npthousandsep{,}
% we want some cells to be coloured
\definecolor{gr}{rgb}{0.95,0.95,1}
\definecolor{firstcol}{rgb}{0.95,0.95,1}
\definecolor{sum}{rgb}{0.9,0.9,0.9}
% automatically round computations to 1 digit
\STautoround{1}
% spreadtab: it takes as argument the normal tabular environment.
\spreadtab{{tabu}{lXn70n70n70}}
\toprule
	&
	& @{year 1}
	& @{year 2}
	& @{year 3}\\\midrule
@\multicolumn2l{{\bf planned publications}}&&&\\
%
% the \rowcolor is not tabu, it is xcolor. 
% Should not use it here.  But it works.
\rowcolor{gr}@\cellcolor{firstcol}&@ journal papers
	& 2
	& 3
	& 5\\
%
\rowcolor{gr}@\cellcolor{firstcol}&@ conference papers
	& 4
	& 6
	& 2\\
%
@\multicolumn2l{{total}}
	& [0,-2]+[0,-1]
	& [0,-2]+[0,-1]
	& [0,-2]+[0,-1]\\\midrule
%
@\multicolumn2l{{\bf number of lines to be typed}}&&&\\
%
\rowcolor{gr}@\cellcolor{firstcol}&@ lines of code
	& 10000
	& 20000
	& 40000\\
%
\rowcolor{gr}@\cellcolor{firstcol}&@ lines of text
	& 2000
	& 3000
	& 4000\\
%
@\multicolumn2l{{total number of lines}}
	& [0,-1]+[0,-2]
	& [0,-1]+[0,-2]
	& [0,-1]+[0,-2]\\\midrule
%
% change colour of next row using tabu
\taburowcolors1{sum .. sum}
@\multicolumn2l{{\bf average number of characters}}
	& 77*[0,-1]
	& 77*[0,-1]
	& 77*[0,-1]\\\midrule
% change back
\taburowcolors1{white .. white}
	%
@\bf employees&&&&\\ 
%
\rowcolor{gr}@\cellcolor{firstcol}&@ number of Ph.D. students
	& 0
	& 1
	& 3\\
%
\rowcolor{gr}@\cellcolor{firstcol}&@ productivity: 2.000 lines per month
	& 2000*[0,-1]*12
	& 2000*[0,-1]*12
	& 2000*[0,-1]*12\\
%
\rowcolor{gr}@\cellcolor{firstcol}&@ number of M.Sc.\ students
	& 2
	& 2
	& 1\\
%
\rowcolor{gr}@\cellcolor{firstcol}&@ productivity: 200 lines per month
	& 200*[0,-1]*12
	& 200*[0,-1]*12
	& 200*[0,-1]*12\\
%
@\multicolumn2l{@{total line productivity}}
	& [0,-3]+[0,-1]
	& [0,-3]+[0,-1]
	& [0,-3]+[0,-1]\\\midrule
%
\taburowcolors1{sum .. sum}
@\multicolumn2l{@{\bf lines still to be written}}
	& [0,-8]-[0,-1]
	& [-1,-8]+[0,-8]-([-1,-1]+[0,-1])
	% in the next line we prevent negative numbers
	& iflt([-2,-8]+[-1,-8]+[0,-8]-([-2,-1]+[-1,-1]+[0,-1]),0,0,
	       [-2,-8]+[-1,-8]+[0,-8]-([-2,-1]+[-1,-1]+[0,-1]))\\\bottomrule
%
\endspreadtab

\end{document}